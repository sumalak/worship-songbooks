\documentclass[17pt]{extarticle}

\usepackage{songs}
%\usepackage[lyric]{songs}  % Вывести только слова.

% Настройки для русского языка (xelatex).
\usepackage{polyglossia}
\setdefaultlanguage[babelshorthands=true]{russian}

% Используемые шрифты (xelatex).
\setmainfont{DejaVuSerif}
\setsansfont{DejaVuSans}
\setmonofont{DejaVuSansMono}

% Убранные параметры, так как используется xelatex (данные параметры нужны только для pdflatex).
%\usepackage[utf8]{inputenc}
%\usepackage[T2A,T1]{fontenc}
%\usepackage[russian]{babel}
%\usepackage{DejaVuSans}

% papersize - размер станицы;
% includefoot - учитывать подвал отдельно;
% hmargin - горизонтальный отступ;
% vmargin - вертикальный отступ;
% foot - высота подвала;
% Для показа краёв полей - параметр showframe.
\usepackage[papersize={210mm,297mm}, includefoot, hmargin=10mm, vmargin=5mm, foot=8mm]{geometry}

%\usepackage[papersize={12.8cm,9.6cm}, hmargin=1cm, vmargin=0cm]{geometry}
%\geometry{tmargin=1cm,bmargin=1cm,lmargin=2cm,rmargin=2cm}

\renewcommand{\lyricfont}{\sffamily}
\renewcommand{\scripturefont}{\sffamily\itshape}
\renewcommand{\printchord}[1]{\sffamily\itshape#1}  % С этой командой аккорды держатся размера текста.
\renewcommand{\stitlefont}{\sffamily\itshape\Large}

\nosongnumbers  % Задача по нумерации переложена на страницы.
%\setlength{\songnumwidth}{2cm}  % Увеличил, иначе сообщения про переполнение \hbox появляются.
%\renewcommand{\printsongnum}[1]{\sffamily\bfseries\large#1}
%\renewcommand{\spngnumbgcolor}{white}

\versesep=5mm plus 3mm minus 5mm  % Разрешил увеличивать и уменьшать отступ после куплетов.

%\pagenumbering{gobble}  % Отмена нумерации страниц.

\songcolumns{0}  % Отключен механизм построения колонок.

\spenalty=-10000  % Заставляет переносится на отдельную страницу.

\DeclareLyricChar{\dots}  % Чтобы между словом и троеточием не ставилось тире из-за близкого аккорда.

\newindex{index}{songs-balzam-1}  % Программу songidx я собрал из исходников отсюда: http://www.ctan.org/tex-archive/macros/latex/contrib/songs
\indexsongsas{index}{\thepage}  % Индексация по номеру страницы.

%\notenamesin{A}{B}{C}{D}{E}{F}{G}  % Замена названий аккордов (по желанию).
%\notenamesout{ЛЯ}{СИ}{ДО}{РЕ}{МИ}{ФА}{СОЛЬ}

% INFO При переносах на другую строку предлоги не должны отрываться от следующего слова. Тире не отрывается от предыдущего слова. Для этого используется жёсткий пробел ~.

% INFO You can explicitly dictate how much of the text following a chord macro is to appear under the chord name by using braces. To exclude text that would normally be drawn under the chord, use a pair of braces that includes the chord macro.

% INFO Для уменьшения текста гимна добавить /small после /beginsong{} (пример в гимне "Что ты сделаешь Богу распятому").

% INFO Сборник хорошо подходит для двухсторонней печати и указанием масштаба 100% или actual size.

\renewcommand{\extendpostlude}{  % Изменяем команду показа копирайта и лицензии, добавляя в конец веб-сайт.
\songcopyright\ \songlicense\unskip
\ www.jesustoday.ru | v0.2.0
}
